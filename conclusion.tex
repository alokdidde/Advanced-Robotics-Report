\chapter{Conclusion}

The agricultural sector is undergoing a transformative phase, driven by the integration of advanced technologies such as robotics, spatial mapping, and multi-modal language models. This report has explored how these technologies, when applied effectively, can revolutionize traditional farming practices and address the pressing challenges faced by modern agriculture, such as climate change, resource scarcity, and the growing global demand for food.

One of the central contributions of this research has been the application of Gaussian splatting for real-time navigation and mapping in agricultural environments. By transforming sparse 3D point clouds into continuous Gaussian distributions, this method provides a precise and efficient representation of the environment. The ability of agricultural robots to navigate through complex terrains with safety and accuracy is greatly enhanced by this technique. Gaussian splatting also supports dynamic real-time updates, allowing robots to continually refine their understanding of the environment as they move through it. This level of precision in navigation is crucial for autonomous farming systems, where the ability to make real-time decisions can lead to more efficient field operations and reduce the margin of error in crop management.

Another major focus of this report has been on the use of convolutional neural networks (CNNs) for disease and pest detection in crops. The CNN-based models trained on agricultural datasets have demonstrated their effectiveness in accurately identifying diseases and pests from plant images. This capability is vital for precision agriculture, as it allows for targeted interventions that can prevent widespread crop damage, reduce the unnecessary use of pesticides, and promote healthier crop growth. By leveraging CNNs, farmers can move away from traditional blanket spraying techniques and instead adopt a more focused approach that conserves resources and minimizes environmental impact.

In addition to these advancements, the integration of Visual Language Action Models (VLAMs) into agricultural robotics represents a significant leap forward. VLAMs allow robots to interpret and respond to natural language instructions in combination with visual data, enabling them to perform complex tasks autonomously. For example, a robot equipped with VLAM capabilities can navigate a field, identify specific crops, and execute actions such as spraying pesticides or applying fertilizers based on spoken commands. This integration of language and vision into robotic control systems enhances the versatility and adaptability of agricultural robots, allowing them to handle a wider range of tasks with minimal human intervention.

While the current system shows significant promise, several areas for improvement have been identified. The following improvements could enhance the performance and efficiency of the system:

\begin{itemize}
    \item \textbf{Onboard Computer:} The system would benefit from the inclusion of an onboard computer with sufficient computational power to run the necessary models directly on the robot. Currently, the processing relies heavily on external systems, which introduces latency and reduces real-time responsiveness. An onboard computer would enhance autonomy and reduce dependency on cloud-based infrastructure, allowing for quicker decision-making and more efficient operations in the field.
    \item \textbf{Wheelbase Optimization:} The robot's current wheelbase design is not optimal for navigating farm environments. \cite{Elsheikh2023} The existing design struggles with uneven terrain and lacks the necessary stability for consistent performance in rough agricultural settings. A track-based system with suspension would provide better traction, stability, and adaptability to the varying conditions found in farms, leading to improved navigation and task execution.
    \item \textbf{Frame Sturdiness:} The robot's frame, currently constructed from aluminum extrusion, has shown weaknesses under the demands of field operation. Replacing the frame with a more robust core material could significantly improve the robot's durability and longevity. A sturdier frame would better withstand the stresses of farm work, especially when carrying heavy equipment or operating in harsh conditions.
\end{itemize}

The implications of these findings are far-reaching. The successful integration of robotics, spatial mapping, and language models into agriculture can lead to significant improvements in productivity, sustainability, and cost-efficiency. Autonomous robots equipped with advanced perception and decision-making capabilities can perform repetitive and labor-intensive tasks, freeing up human labor for more strategic activities. Moreover, by optimizing the use of inputs such as water, fertilizers, and pesticides, these technologies can contribute to more sustainable farming practices that are better aligned with environmental conservation goals.

Despite the promising results, this research also highlights areas where further exploration is needed. The future of agricultural robotics will depend on continuous improvements in several key areas. First, advancements in real-time data processing and the development of more robust algorithms will be essential to ensure that robots can operate effectively in diverse and unpredictable agricultural environments. Second, the availability of larger and more diverse datasets for disease and pest detection will be critical in improving the accuracy and generalization capabilities of CNN models.\cite{cieslak2024generatingdiverseagriculturaldata} As agricultural environments vary widely across regions and climates, models must be trained on data that reflect this diversity to be effective on a global scale.

Furthermore, the generalization capabilities of Visual Language Action Models must be enhanced to handle more complex and nuanced instructions, especially in multilingual and culturally diverse farming contexts. Fine-tuning these models for specific agricultural tasks while ensuring they maintain their adaptability to new tasks is an ongoing challenge. Additionally, as these technologies continue to develop, ethical considerations such as data privacy, the impact on rural employment, and the equitable distribution of technological benefits must be addressed to ensure that the advancements in agricultural robotics contribute positively to society as a whole.

In conclusion, the combination of robotics, spatial mapping, and multi-modal language models holds immense potential for the future of agriculture. By continuing to innovate in these areas, we can create more efficient, resilient, and sustainable farming systems that are capable of meeting the demands of a growing global population. This research marks an important step toward realizing that vision, but there is still much work to be done. As we move forward, the integration of cutting-edge technologies into agriculture will require a collaborative effort from researchers, engineers, farmers, and policymakers to ensure that these advancements are implemented in ways that maximize their benefits for both people and the planet.
