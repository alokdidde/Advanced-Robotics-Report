\chapter{Introduction}

\section{Motivation and Background}
Agriculture, one of the world's oldest industries, faces increasing pressure to meet the growing global demand for food while contending with environmental challenges such as climate change, soil degradation, and water scarcity. \cite{duckett2018agriculturalroboticsfuturerobotic} Traditional farming practices are no longer sufficient to address these challenges. As a result, technological innovations in robotics, artificial intelligence (AI), and data analytics are rapidly transforming agriculture into a more efficient, precise, and sustainable industry. 

This report focuses on the application of advanced technologies such as robotics, spatial mapping, and multi-modal language models in crop management. These innovations have the potential to enhance agricultural efficiency by providing detailed environmental mapping, precision in crop health monitoring, and autonomous decision-making in real-time.

\section{Objectives and Scope}
The primary objective of this exercise is to explore the integration of cutting-edge technologies in agriculture to optimize crop management practices. The following key technologies have been investigated:
\begin{itemize}
    \item \textbf{Gaussian Splatting for Real-Time Navigation and Mapping:} This technique transforms sparse 3D point clouds into continuous Gaussian distributions, enabling precise navigation and environmental understanding for agricultural robots.\cite{chen2024splatnavsaferealtimerobot}
    \item \textbf{Convolutional Neural Networks (CNNs) for Disease and Pest Detection:} Deep learning models are employed to identify diseases and pests from plant images, facilitating targeted interventions.\cite{10353343}
    \item \textbf{Visual Language Action Models (VLAMs):} These models integrate visual perception and language understanding to control agricultural robots, enabling them to perform complex tasks such as spraying pesticides or navigating through fields based on natural language instructions.
\end{itemize}

This aims to demonstrate how these technologies can be combined into a cohesive agricultural robot system that autonomously navigates fields, detects crop health issues, and takes appropriate actions, thereby improving productivity and reducing resource use.\cite{sitokonstantinou2024causalmachinelearningsustainable}


\section{Overview of the Report}
The structure of the report is as follows:
\begin{itemize}
    \item \textbf{Chapter 2: Gaussian Splatting for Navigation and Mapping} - This chapter explains how Gaussian Splatting is applied to enable real-time navigation and mapping in agricultural environments, focusing on both theoretical aspects and practical implementations.
    \item \textbf{Chapter 3: Image-Based Disease and Pest Detection} - This chapter delves into the use of convolutional neural networks for detecting diseases and pests in crops, highlighting the dataset, model architecture, and performance evaluation.
    \item \textbf{Chapter 4: Visual Language Action Models (VLAMs) for Agricultural Robotics} - This chapter introduces the concept of VLAMs and their application in controlling agricultural robots through vision and language inputs.
    \item \textbf{Chapter 5: Agricultural Robot System} - This chapter describes the design, construction, and operation of an agricultural robot that autonomously navigates fields, detects crop issues, and takes corrective actions.
    \item \textbf{Chapter 6: Conclusion and Future Work} - The final chapter summarizes the key findings and discusses the potential future directions for research and development in agricultural robotics.
\end{itemize}



